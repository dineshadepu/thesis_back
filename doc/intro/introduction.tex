\newcommand{\etas}{\ensuremath{\eta_{\mathrm{s}}}}


\chapter{Introduction}
Physically based animations have got a quite attention these days. By simulating
the situations close to real life could be very useful in replacing the hand
drawn animations in movies. For an example, the hair simulation is too much work
while a character is running. This could be replaced with the physical
simulation done on fast computers. Similarly interests are flag simulation, can
crushing, blood flow and many more.

Among these, one interesting phenomena is rigid bodies colliding in a fluid
flow.  This modelling can be very much useful in simulating collision of ships
in an ocean.  This problem has two phases, fluid and solid. Where each phases is
interacting with other two phases. For describing the fluids numerically one can
choose Eulerian or a Lagrangian model. Either description has advantages and
disadvantages. Eulerian method the data is taken from a fixed points, where as
in Lagrangian approach, the fluid is discretized into particles, and the data
extracted from those particles as they move. Some of the major advantages of
Eulerian (grid based) schemes is a quick solution of incompressibility can be
found, parallelization is easy. But modelling breaking waves, droplet formation
requires it to use very high resolution grid. Free surface flows modelling is
computationally intensive in grid based schemes. Where as Lagrangian based
schemes are able to capture such tiny details effectively, with less
computational cost. However there are few drawbacks in Lagrangian based schemes.
To overcome disadvantages from Lagrangian scheme, semi-Lagrangian schemes have
proposed.

Smoothed particle hydrodynamics (SPH), a numerical tool of Lagrangian type is
used in the present work to simulate the fluid flow. In SPH it is
computationally expensive to solve for a pressure Poisson equation. In stead the
fluid is assumed to be weakly compressible and the change in density is to
pressure is related with a state equation. For very less density fluctuations,
the speed of sound in the fluid has to be chosen to be very high, which results
in a very small time steps.

To model interaction between the solid bodies Discrete Element Method (DEM) is
used. Given rigid body is discretized into small particles, whose relative
positions don��t change with time, and the interaction force with other rigid
bodies is evaluated using DEM\,. Finally the interaction between the fluid and
solid is evaluated using the pressure from fluid on solid in SPH form.

%%


%%% Local Variables:
%%% mode: latex
%%% TeX-master: "../mainrep"
%%% End:
