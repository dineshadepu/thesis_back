
\newcommand{\etas}{\ensuremath{\eta_{\mathrm{s}}}}


\chapter{Introduction}


This document contains commonly used essential templates to write a
\LaTeX\ document. This document is to be used along with the files and
folders provided. Writing a \LaTeX\ document is very simple.  Often
students need only very simple constructs.  This document shows
certain essential features that almost all technical report writing
requires. Please consult the PDF file for the output of the document,
and then look at the corresponding \LaTeX\ file to reproduce it.  The
document illustrates the following constructs
\begin{itemize}
\item Unnumbered and numbered Lists
\item Equations
\item Defining short macros for frequently used symbols
\item Bibliography
\item Figures
\item Tables
\end{itemize}

The normal procedure for compiling a \LaTeX\ document that contains
bibliographic entries is to follow the following steps
\begin{enumerate}
\item \verb|pdflatex mainrep|
\item \verb|bibtex mainrep|
\item \verb|pdflatex mainrep|
\item \verb|pdflatex mainrep|
\end{enumerate}
In the above example \verb|mainrep| is the main \LaTeX\ file.


\section{First section of this chapter}

This is the first chapter, which resides in a directory (folder)
intro. Each chapter can contain \verb|section|, \verb|subsection|
and so on.

\subsection{Equations and Math symbols}


Equations should be set in a separate mode.  For details on getting
various types of aligned equations, consult the \AmS-\LaTeX\
documentation \verb|amsldoc.pdf|. Simple equations are set as
\begin{equation}
\label{eq:sinx}
\int \mathrm{d}x \; \cos x =  \sin x
\end{equation}
Equation~\eqref{eq:sinx} is the integral of the cosine
function. Mathematical symbols must always be put inside \verb|$$|,
when they appear outside a math environment (such as \verb|equation|,
\verb|align|, \verb|gather|, etc).  The symbol ``ex'' must be written as
$x$ and not as x.  

Another commonly used construct for equations is the \verb|align|
environment to align several equations along a vertical line. It is
usually the $=$ sign across which the alignment is done.  The
point of alignment for each equation is specified using the ampersand symbol 
\begin{align}
a &= b  \\
a + e + f + g & = m + n + z \\
x + 2 & = x^{3} + 3 x^{2} + 2 x + 5
\end{align}

\subsection{Commonly used Symbols}
For mathematical symbols it is very convenient to define frequently
used symbols as a short macro. For example if you are to be using the
symbol $\eta_{\mathrm{s}}$ frequently it is convenient to define it in
as:\\
\verb|\newcommand{\etas}{\ensuremath{\eta_{\mathrm{s}}}}| \\
in the preamble and to simply refer it to in the text as \etas\ or in
a mathematical equation as $\etas = \eta \, ( 1 + \phi)$.
%%
%

\section{How to write nomenclature} 

\subsection{General guidelines:}
\begin{enumerate}	
	\item Use \verb|\nomenclature[prefix]{symbol}{description}| for symbols, the best place for this command is immediately after you introduce the symbol for the first time
	\item Shorten the long command:\\ \verb|\newcommand{\nm}[2]{\nomenclature{#1}{#2}}|
	\item Create compiler for nomenclature with the given code: \\
	\textbf{makeindex \%.nlo -s nomencl.ist -o \%.nls -t \%.nlg }\\
	For TeXstudio: go to options > build > user command > write- `user1: Nomenclature' amd paste the above code\\
	For compiling the nomenclature: go to tools > user > Nomenclature	
\end{enumerate}	

\subsection{Grouped nomenclature}
\begin{enumerate}
\item For acronyms, use:\\
 \verb|\nmA[sorting letter]{symbol}{descritpon}|
\item For roman symbols, use:\\
\verb|\nmR[sorting letter]{symbol}{descritpon}|
\item For greek symbols, use:\\
 \verb|\nmG[sorting letter]{symbol}{descritpon}|
 \item For superscripts, use:\\
 \verb|\nmS[sorting letter]{symbol}{descritpon}|
 \item For subscripts, use:\\
 \verb|\nms[sorting letter]{symbol}{descritpon}| 
 \item For any other symbol, use:\\
 \verb|\nmX[sorting letter]{symbol}{descritpon}|\\
 Name of other symbols can be changed with \verb|\OtherSym{Name of symbols}|
\end{enumerate}
%%
\subsection{Some examples}
\begin{enumerate}
\item \verb|\nmA[FF]{FFA}{Free fatty acid}|
\item \verb|\nmA[AO]{AOR}{Angle of repose}|
\item \verb|\nmR[Ra]{$R$}{Radius of circle}|
\item \verb|\nmR[ra]{$r$}{Intrinsic length}|
\item \verb|\nmR[Gr]{$G_\mathrm{r}$}{Gravity}|
\item \verb|\nmG[al]{$\alpha_{\mathrm{a}}$}{Angular acceleration}|
\item \verb|\nmG[et]{$\eta$}{Viscosity}|
\item \verb|\nmG[be]{$\beta$}{Shape factor}|
\item \verb|\nmS[v]{$v$}{Vapor phase}|
\item \verb|\nmS[g]{$g$}{Gas phase}|
\item \verb|\nms[i]{$i$}{Indices}|
\item \verb|\nms[x]{$x$}{Variable in x-direction}|
\item \verb|\nmX[f]{foo}{foo|
\end{enumerate} 

\nmA[FF]{FFA}{Free fatty acid}
\nmA[AO]{AOR}{Angle of repose}


\nmR[Ra]{$R$}{Radius of circle}
\nmR[ra]{$r$}{Intrinsic length}
%\nmR[Gr]{$G_\mathrm{r}$}{Gravity}


\nmG[al]{$\alpha_{\mathrm{a}}$}{Angular acceleration}
\nmG[et]{$\eta$}{Viscosity}
%\nmG[be]{$\beta$}{Shape factor}


\nmS[v]{$v$}{Vapor phase}
\nmS[g]{$g$}{Gas phase}


\nms[i]{$i$}{Indices}
\nms[x]{$x$}{Variable in x-direction}


\nmX[f]{foo}{foo}


%%


%%% Local Variables: 
%%% mode: latex
%%% TeX-master: "../mainrep"
%%% End: 
