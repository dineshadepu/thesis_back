\chapter{Fluid simulation using Smoothed Particle Hydrodynamics}
\label{cha:fluid-simul-using}

Mass conservation equation for a fluid particle can be written as

\begin{equation}
  \label{eq:continuity}
  \frac{D\rho}{Dt} = - \rho \; \nabla \cdot \vec{u}
\end{equation}



Momentum conservation equation for a fluid particle can be written as

\begin{equation}
  \label{eq:momentum}
  \frac{D\vec{u}}{Dt} = - \frac{\nabla p}{\rho} + \nu
  \nabla^2{\vec{u}} + \vec{g}
\end{equation}



Any function can be approximated by using its neighbouring points by Delta
function. Which later can be extended to a kernel function W.



\begin{equation}
  \label{eq:f_app}
  A(x_i) = \sum_i \frac{m_j}{\rho_j} A(x_j) W_{ij}
\end{equation}


Approximation of divergence of a function is necessary to approximate
continuity equation \eqref{eq:continuity}.  One simple way of approximating would
be directly by using the function approximation, and extending it to divergence.

\begin{equation*}
  \nabla \cdot \vec{A}(x_i) = \sum_i \frac{m_j}{\rho_j}
  \vec{A}(x_j) \cdot \nabla_i W_{ij}
\end{equation*}

A better approximation would be done by using density from \textit{First golden rule
of SPH} \cite{monaghan-1992-smoot-partic-hydrod}.  From basic vector identity we
have

\begin{equation*}
  \nabla\cdot(f \vec{u}) = \vec{u} \cdot
  \nabla{f} + f \; \nabla \cdot \vec{u}
\end{equation*}

From which the divergence of a vector is computed as

\begin{equation*}
  \nabla \cdot \vec{u} = \frac{1}{f} \Big[\nabla\cdot(f
  \vec{u}) - \vec{u} \cdot \nabla{f} \Big]
\end{equation*}

In our case $f$ is $\rho$, and $u$ is velocity.  With particle
approximation,

\begin{equation*}
  \nabla \cdot \vec{u}_i = \frac{1}{\rho_i}
  \Big[\nabla\cdot(f_i \vec{u}_i) - \vec{u}_i \cdot \nabla{\rho}_i
  \Big]
\end{equation*}

Applying SPH approximation would be

\begin{equation*}
  \nabla \cdot \vec{u}_i = \frac{1}{\rho_i} \Big[
  \sum_j \frac{m_j}{\rho_j} \rho_j \vec{u}_j \cdot \nabla_i W_{ij} - \sum_j
  \frac{m_j}{\rho_j} \rho_j \vec{u}_i \cdot \nabla_i W_{ij} \Big]
\end{equation*}

Which gives us divergence approximation in terms of density

\begin{equation}
  \label{eq:divergence_sph}
  \nabla \cdot \vec{u}_i = \frac{1}{\rho_i}
  \Big[ \sum_j {m_j} (\vec{u}_j - \vec{u}_i) \cdot \nabla_i W_{ij}
  \Big]
\end{equation}


Since in momentum equation ref:eq:momentum we have pressure force as gradient
of pressure, we need to derive SPH approximation of such gradient. From naive
approximation we have,


\begin{equation}
  \label{eq:df_app}
   \nabla A(x_i) = \sum_i \frac{m_j}{\rho_j} A(x_j) \nabla_i W_{ij}
\end{equation}


A better approximation would be done by using density from \textit{First golden rule
of SPH} \cite{monaghan-1992-smoot-partic-hydrod}.  From basic vector identities we
have

\begin{align*}
  \nabla \bigg( \frac{f}{g} \bigg) &= f\; \nabla \bigg(\frac{1}{g} \bigg) + \frac{1}{g} \nabla f \\
  \frac{1}{g} \nabla f &= \nabla \bigg( \frac{f}{g} \bigg) - f\; \nabla \bigg( \frac{1}{g} \bigg) \\
  \frac{1}{g} \nabla f &= \nabla \bigg( \frac{f}{g} \bigg) + \frac{f}{g^2} \; \nabla g
\end{align*}

in our case $f$ is pressure (p) and $g$ is density ($\rho$), and
approximating it for a particle,


\begin{equation*}
  \frac{1}{\rho_i} \nabla p_i = \nabla \big(\frac{p_i}{\rho_i} \big) + \frac{p_i}{\rho_i^2} \; \nabla \rho
\end{equation*}

Approximating it using SPH, we have

\begin{equation*}
  \frac{1}{\rho_i} \nabla p_i = \sum_j \frac{m_j}{\rho_j} \;
  \frac{p_j}{\rho_j} \nabla_i W_{ij} + \frac{p_i}{\rho_i^2} \; \sum_j
  \frac{m_j}{\rho_j} \; \rho_j \nabla_i W_{ij}
\end{equation*}

Which gives us gradient of $p$ in terms of density as


\begin{equation}
  \label{eq:df_sph}% \frac{1}{\rho_i} \nabla p_i = \sum_j m_j \bigg(
  \frac{p_i}{\rho_i^2} + \frac{p_j}{\rho_j^2} \bigg)\;\nabla_i W_{ij}
\end{equation}


One of ways to solve for incompressible fluids in SPH is using PCISPH. In
PCISPH, if the density at the next time step in more compressible than the
prescribed percent, the the pressure at the current time is adjusted to nullify
it to the prescribed limit.

The density is computed using summation density.

\begin{equation}
  \label{eq:summation_density}
  \rho_i(t) = \sum_j m_j W(\vec{x}_i(t) - \vec{x}_j(t))
\end{equation}

Now, density at time $t + 1$ in terms of summation density can be written as,
and assuming equal mass(m) for each particle,
\begin{align*}
  \rho_i(t+1) &= m \sum_j W(\vec{x}_i(t+1) -
                \vec{x}_j(t+1)) \\
              &= m \sum_j W(\vec{x}_i(t) + \Delta
                \vec{x}_i(t) - \vec{x}_j(t) - \Delta \vec{x}_j(t)) \\
              &= m \sum_j W(\vec{d}_{ij}(t) + \Delta \vec{d}_{ij}(t))
\end{align*}

Where $\vec{d}_{ij}$ is $\vec{x}_i(t) - \vec{x}_j(t)$ and
$\Delta \vec{d}_{ij}$ is $\Delta\vec{x}_i(t) -
\Delta\vec{x}_j(t)$


Applying Taylor series approximation to kernel, we have

\begin{align*}
  \rho_i(t+1) &= m \sum_j W(\vec{d}_{ij}(t)) + m \sum_j \nabla_i W(\vec{d}_{ij}(t)) \cdot \Delta \vec{d}_{ij}(t) \\
              &= m \sum_j W(\vec{x}_i(t) - \vec{x}_j(t)) + m \sum_j \nabla_i W(\vec{x}_i(t) -
                \vec{x}_j(t)) \cdot (\Delta\vec{x}_i(t) - \Delta\vec{x}_j(t))\\
\end{align*}

The first term on the right hand side is the density at time $t$ and the
second term is the change at time $t$.

\begin{equation*}
  \rho_i(t+1) = \rho_i(t) + \Delta \rho_i(t)
\end{equation*}

The change in density can be written as

\begin{equation}
  \label{eq:density_difference}
  \Delta \rho_i(t) = m \big(\Delta\vec{x}_i(t) \cdot \sum_j \nabla_i W_{ij} - \sum_j \nabla_i W_{ij} \cdot \Delta\vec{x}_j(t) \big)
\end{equation}

From the integrator scheme (Leap-Frog) we can estimate the change in position
of $x_i$ as


\begin{equation*}
  \label{eq:Delta_x_i}
  \Delta \vec{x}_i = \Delta t^2
  \frac{\vec{F}_i^p}{m}
\end{equation*}

Since pressure force per unit mass is given by


\begin{equation*} \frac{\vec{F}_i^p}{m} = - \frac{\nabla p_i}{\rho_i}
\end{equation*}

the right hand side can be written in terms of particle approximation of
gradient \ref{eq:df_sph} as

\begin{equation*}
  \frac{\vec{F}_i^p}{m} = - \sum_j m_j \bigg(
  \frac{p_i}{\rho_i^2} + \frac{p_j}{\rho_j^2} \bigg)\;\nabla_i W_{ij}
\end{equation*}

by assuming equal mass for every particle, this can be reformulated as

\begin{equation*}
  \frac{\vec{F}_i^p}{m} = - m \sum_j \bigg(
  \frac{p_i}{\rho_i^2} + \frac{p_j}{\rho_j^2} \bigg)\;\nabla_i W_{ij}
\end{equation*}

Which can be substituted in \ref{eq:Delta_x_i}, and can be solved for change in
position of $x_i$

\begin{equation*}
  \Delta \vec{x}_i = - \Delta t^2 \; m \; \sum_j \bigg(
  \frac{p_i}{\rho_i^2} + \frac{p_j}{\rho_j^2} \bigg)\;\nabla_i W_{ij}
\end{equation*}

Further assuming the pressure of neighbourhood particles same as particle i,
and density for all the particles to be same i.e., $\rho_0$ we have

\begin{equation*}
  \Delta \vec{x}_i = - \Delta t^2 \; m \; \frac{2 \;
    \widetilde{p}_i}{\rho_0^2} \sum_j \nabla_i W_{ij}
\end{equation*}

Force acting on particle $i$ due to particle $j$ can be written as

\begin{equation*}
  \vec{F}_j^p = m^2 \bigg( \frac{p_i}{\rho_0^2} +
  \frac{p_j}{\rho_0^2} \bigg)\;\nabla_i W_{ij}
\end{equation*}

Note that the sign is reversed as there will be equal and opposite forces on
both the particles and the gradient is still evaluated with respect to particle
$i$. Due to such force, the change in position of particle at $j$ is given by

\begin{equation*}
  \Delta \vec{x}_j = \Delta t^2 \; m \; \frac{2 \;
    \widetilde{p}_i}{\rho_0^2} \nabla_i W_{ij}
\end{equation*}

Now substitute particle $i$ and $j$ position change in density difference
equation \ref{eq:density_difference}, we have,

\begin{align*}
  \Delta \rho_i(t) &= m \bigg( - \Delta t^2 \; m \; \frac{2 \;
                                  \widetilde{p}_i}{\rho_0^2} \sum_j \nabla_i W_{ij} \cdot \sum_j \nabla_i W_{ij} -
                                  \sum_j \nabla_i W_{ij} \cdot \Delta t^2 \; m \; \frac{2 \;
                     \widetilde{p}_i}{\rho_0^2} \nabla_i W_{ij}\bigg)\\
                   &= m^2 \Delta t^2 \; \frac{2 \; \widetilde{p}_i}{\rho_0^2} \bigg(-
                     \sum_j \nabla_i W_{ij} \cdot \sum_j \nabla_i W_{ij} - \sum_j \nabla_i W_{ij} \cdot \nabla_i W_{ij} \bigg)
\end{align*}

Upon solving for pressure $\widetilde{p}$, we get

\begin{align*}
  \widetilde{p}_i &= \frac{\Delta \rho_i(t)}{\beta \; \bigg(-
                                 \sum_j \nabla_i W_{ij} \cdot \sum_j \nabla_i W_{ij} - \sum_j \nabla_i W_{ij}
                                 \cdot \nabla_i W_{ij} \bigg)}
\end{align*}

Where $\beta=\frac{2\,m^2 \Delta t^2}{\rho_0^2}$. To produce a density change
of $\Delta \rho$ we need pressure $\widetilde{p}_i$ as in above equation. But to
nullify a density change of $\rho_{err_i} = \rho_i^* - \rho_0$ we need pressure
$\widetilde{p}_i$ as

\begin{align*}
  \widetilde{p}_i &= \frac{-\rho_{err_i}}{\beta \; \bigg(- \sum_j
                    \nabla_i W_{ij} \cdot \sum_j \nabla_i W_{ij} - \sum_j \nabla_i W_{ij} \cdot
                    \nabla_i W_{ij} \bigg)}\\
  \widetilde{p}_i &= \delta \; \rho_{err_i}
\end{align*}

where $\delta$ is given by

\begin{equation}
  \label{eq:pcisph_delta}
  \delta = \frac{-1}{\beta \; \bigg(- \sum_j \nabla_i
    W_{ij} \cdot \sum_j \nabla_i W_{ij} - \sum_j \nabla_i W_{ij} \cdot \nabla_i
    W_{ij} \bigg)}
\end{equation}

%%% Local Variables:
%%% mode: latex
%%% TeX-master: "../mainrep"
%%% End:
